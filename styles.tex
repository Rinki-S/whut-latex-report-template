% styles.tex - 样式和格式定义
\usepackage[fontset=none]{ctex}
\usepackage{graphicx}
\usepackage{amsmath}
\usepackage[left=3cm, right=3cm, top=2cm, bottom=1.5cm]{geometry}
\usepackage{tabularray}
\usepackage{setspace}
\usepackage{calc}
\usepackage{tikz}
\usepackage{tcolorbox}
\usepackage{enumitem}
\usepackage[ruled, vlined]{algorithm2e}
\usepackage{listings}
\usepackage[dvipsnames]{xcolor}

\tcbuselibrary{breakable}

% 设置代码高亮
\lstset{
  basicstyle=\tt\small,
  breaklines=true,
  frame=single,
  language=C++,
  keywordstyle=\bfseries\color{NavyBlue},
  columns=flexible,
  numbers=left,
  numbersep=-1.5em,
  commentstyle=\itshape\color{gray},
  stringstyle=\bfseries\color{PineGreen},
  numberstyle=\footnotesize
}

% 算法环境本地化
\renewcommand{\algorithmcfname}{算法}
\SetKwInput{KwIn}{输入}
\SetKwInput{KwOut}{输出}

% 字体设置
\setCJKmainfont{思源宋体 CN}
\setCJKsansfont{思源黑体 CN}
\setCJKmonofont{Sarasa Mono SC}
\setmainfont{Times New Roman}
\setsansfont{思源黑体 CN}
\setmonofont{Courier Prime}

% 通用样式命令
\newcommand{\tcstyle}{\zihao{5}\sffamily}
\newcommand{\tchstyle}{\zihao{5}\sffamily\bfseries}
\newcommand{\blankline}{\vspace{\baselineskip}}
\newcommand{\enableprintmode}{\printmodetrue}
\newcommand{\disableprintmode}{\printmodefalse}

% 标题命令
\newcounter{reportpart}
\newcommand{\hone}[1]{
  \stepcounter{reportpart}
  {\zihao{-3}\sffamily\bfseries\noindent 第\chinese{reportpart}部分:#1}\par
  \vspace{0.5em}
}

\newcounter{reportsect}[reportpart]
\newcommand{\htwo}[1]{
  \stepcounter{reportsect}
  {\zihao{4}\bfseries\noindent \chinese{reportsect}、~~#1}\par
  \vspace{0.3em}
}

\newcounter{reportsubsect}[reportsect]
\newcommand{\hthree}[1]{
  \stepcounter{reportsubsect}
  {\zihao{-4}\bfseries\noindent \arabic{reportsubsect}.~~#1}\par
  \vspace{0.2em}
}

\newcounter{reportitem}[reportsubsect]
\newcommand{\hfour}[1]{
  \stepcounter{reportitem}
  {\zihao{5}(\arabic{reportitem})~~#1}\par
  \vspace{0.1em}
}

% 下划线命令
\newcommand{\tikzunderline}[2][6.655cm]{%
  \begingroup
  \sbox0{\parbox[t]{#1}{\centering #2}}%
  \begin{tikzpicture}[baseline=(X.base)]
    \node[anchor=base,inner sep=0pt,outer sep=0pt,text width=#1,align=center] (X) {\usebox0};
    \draw[thick] ([yshift=-7pt]X.south west) -- ([yshift=-7pt]X.south east);
  \end{tikzpicture}%
  \endgroup
}